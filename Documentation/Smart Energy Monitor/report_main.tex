\documentclass{crceitthesis-3}
\usepackage{longtable} %for long tables to be fitted on multiple pages
\usepackage{ragged2e}
%\usepackage{color}
\usepackage{graphicx}
\usepackage{varioref}
\usepackage{epsfig}
\usepackage{times}
\usepackage[section]{placeins} % to keep floats(figures) in the section in which they were issued
\usepackage{amsmath}
\usepackage{amssymb}
\usepackage{url}
\usepackage{multirow}
\usepackage[T1]{fontenc}
\usepackage{ascii}
\usepackage{nomencl}
\usepackage{lscape}
\usepackage{longtable,tabu}
\usepackage{lmodern}
\usepackage{lipsum}
\usepackage[tight,footnotesize]{subfigure}
\usepackage{cite}
\usepackage[acronym]{glossaries} 
\usepackage{fancybox}
\usepackage{algorithmic}
\usepackage[boxed]{algorithm}
\usepackage[compact]{titlesec}
\usepackage{gensymb}
\usepackage{multicol}
%----------For code listing---------------------------------------------------
\usepackage{listings} % Required for inserting code snippets
\usepackage[usenames,dvipsnames]{color} % Required for specifying custom colors and referring to colors by name

\definecolor{DarkGreen}{rgb}{0.0,0.4,0.0} % Comment color
\definecolor{highlight}{RGB}{255,251,204} % Code highlight color

\lstdefinestyle{Style1}{ % Define a style for your code snippet, multiple definitions can be made if, for example, you wish to insert multiple code snippets using different programming languages into one document
language=Scilab, % Detects keywords, comments, strings, functions, etc for the language specified
backgroundcolor=\color{highlight}, % Set the background color for the snippet - useful for highlighting
basicstyle=\footnotesize\ttfamily, % The default font size and style of the code
breakatwhitespace=false, % If true, only allows line breaks at white space
breaklines=true, % Automatic line breaking (prevents code from protruding outside the box)
captionpos=b, % Sets the caption position: b for bottom; t for top
commentstyle=\usefont{T1}{pcr}{m}{sl}\color{DarkGreen}, % Style of comments within the code - dark green courier font
deletekeywords={}, % If you want to delete any keywords from the current language separate them by commas
%escapeinside={\%}, % This allows you to escape to LaTeX using the character in the bracket
firstnumber=1, % Line numbers begin at line 1
frame=single, % Frame around the code box, value can be: none, leftline, topline, bottomline, lines, single, shadowbox
frameround=tttt, % Rounds the corners of the frame for the top left, top right, bottom left and bottom right positions
keywordstyle=\color{Blue}\bf, % Functions are bold and blue
morekeywords={}, % Add any functions no included by default here separated by commas
numbers=left, % Location of line numbers, can take the values of: none, left, right
numbersep=10pt, % Distance of line numbers from the code box
numberstyle=\tiny\color{Gray}, % Style used for line numbers
rulecolor=\color{black}, % Frame border color
showstringspaces=false, % Don't put marks in string spaces
showtabs=false, % Display tabs in the code as lines
stepnumber=5, % The step distance between line numbers, i.e. how often will lines be numbered
stringstyle=\color{Purple}, % Strings are purple
tabsize=2, % Number of spaces per tab in the code
}

% Create a command to cleanly insert a snippet with the style above anywhere in the document
\newcommand{\insertcode}[2]{\begin{itemize}\item[]\lstinputlisting[caption=#2,label=#1,style=Style1]{#1}\end{itemize}} % The first argument is the script location/filename and the second is a caption for the listing

%--------------------------------------End of code listing-------------------

\titlespacing{\section}{0pt}{2ex}{1ex}
\titlespacing{\subsection}{0pt}{1ex}{0ex}
\titlespacing{\subsubsection}{0pt}{0.5ex}{0ex}

\usepackage[compact]{titlesec}
\titleformat{\chapter}[display]
  {\normalfont\huge\bfseries\centering}{\chaptertitlename\ \thechapter}{20pt}{\LARGE}

%\usepackage[toc,page]{appendix}

\makeglossaries
\makenomenclature

\newglossaryentry{wsn}{name=WSN,description=Wireless Sensor Network}
\newglossaryentry{manet}{name=MANET,description=Mobile Ad hoc NETwork}
\newglossaryentry{gps}{name=GPS,description=Global Positioning System}
\newglossaryentry{spin}{name=SPIN,description=Sensor Protocols for Information via Negotiation}
\newglossaryentry{dc}{name=DC,description=Data Centric}
\newglossaryentry{bs}{name=BS,description=Base Station}
\newglossaryentry{mcfa}{name=MFCA,description=Minimum Cost Forwarding Algorithm}
\newglossaryentry{acquire}{name=ACQUIRE,description=Active Qwery Forwarding in Sensor Networks}
\newglossaryentry{leach}{name=LEACH,description=Low Energy Adaptive Clustering Hierarchical}
\newglossaryentry{leachc}{name=LEACH-C,description=LEACH Centralized}
\newglossaryentry{pegasis}{name=PEGASIS,description=Power Efficient Gathering in Sensor Information Systems}
\newglossaryentry{teen}{name=TEEN,description=Threshold-Sensitive Energy Efficient SensormNetwork Protocol}
\newglossaryentry{heed}{name=HEED,description=Hybrid Energy-Efficient Distributed}
\newglossaryentry{gaf}{name=GAF,description=Geographic Adaptive Fidelity}
\newglossaryentry{gear}{name=GEAR,description=Geographic and Energy Aware Routing}
\newglossaryentry{sleachc}{name=sLEACH-C,description=Solar-aware LEACH-centralized extension}
\newglossaryentry{group}{name=GROUP,description=Genetic algorithm inspired ROUting Protocol}
\newglossaryentry{aco}{name=ACO,description=Genetic algorithm inspired ROUting Protocol}
\newglossaryentry{pso}{name=PSO,description=Particle swarm optimization}

\newglossaryentry{aodv}{name=AODV,description=Ad hoc On-demand Distance Vector}
\newglossaryentry{dsr}{name=DSR,description=Dynamic Source Routing}
\newglossaryentry{rfc}{name=RFC,description=Request For Comments}
\newglossaryentry{ara}{name=ARA,description=Ant-Inspired Routing Algorithm}
\newglossaryentry{ec}{name=EC,description=Evolutionary Computation}
\newglossaryentry{ga}{name=GA,description=Genetic Algorithms}
\newglossaryentry{ils}{name=ILS,description=Iterated Local Search}
\newglossaryentry{sa}{name=SA,description=Simulated Annealing}
\newglossaryentry{ts}{name=TS,description=Tabu Search}
\newglossaryentry{si}{name=SI,description=Swarm Intelligence}
\newglossaryentry{as}{name=AS,description=Ant System}
\newglossaryentry{eas}{name=EAS,description=Elitist strategy for Ant System}
\newglossaryentry{tsp}{name=TSP,description=Traveling Salesman Problem}
\newglossaryentry{so}{name=SO,description=Self-Organization}
%%%%%%%%%%%%%%%%%%%%%%%%%%%%%MAIN PART OF THE REPORT%%%%%%%%%%%%%%%%%%%%%%%%%%%%%%%
\begin {document}
\graphicspath{{./}{images/}}
\headheight-8pt % to adjust white space.
\headsep -12pt %
\footskip 18mm %
% prelude.tex
%   - titlepage
%   - dedication (optional)
%   - approval sheet
%   - table of contents, list of tables and list of figures
%   - abstract
%============================================================================


\clearpage\pagenumbering{roman}  % This makes the page numbers Roman (i, ii, etc)


% TITLE PAGE
%   - define \title{} \author{} \date{}
\title{Smart Energy Monitor}
\stua{Shabbir Ahmed}
\stub{Dylan Dcruz}
\stuc{Jonathan Pereira}

\date{\large October 26, 2017} %\today

%  - Roll number, required for title page, approval sheet, and
%    certificate of course work 

\rollnuma{7173}
\rollnumb{7304}
\rollnumc{7337}

%   - The default degree is ``Doctor of Philosophy''
%     (unless the document style msthesis is specified
%      and then the default degree is ``Batchlor of Engineering'')
%     Degree can be changed using the command \mudegree{}
\mudegree{Bachelor of Engineering}

%   - The default report type is preliminary report.
%      * for a PhD thesis, specify \thesis
%\thesis
%      * for a M.Tech./M.Phil./M.Des./M.S. dissertation, specify \dissertation
\project
%      * for a DIIT/B.Tech./M.Sc.project report, specify \project
%\project
%      * for any other type, use  \reporttype{}
%\reporttype{ReportType}

%   - The default department is ``Unknown Department''
%     The department can be changed using the command \department{}
\department{DEPARTMENT OF ELECTRONICS}

%\section{Graphics}
%    - Set the guide's name
\setguide{Prof. Jayen Modi}
%    - Set the coguide's name (if you have one)
%\setcoguide{PPP}
%    - Set external guide (if you have one)
%\setexguide{Prof External Guide}

%   - once the above are defined, use \maketitle to generate the titlepage
\maketitle

%--------------------------------------------------------------------%
% DEDICATION
%   Dedications, if any, must be first page after title page.
\begin{dedication}
 \bf \it 
%I appreciate and am very thankful for their continued motivation and support. 
\end{dedication}

%--------------------------------------------------------------------%
% APPROVAL SHEET
%   - for final thesis, you need Approval Sheet. So, uncomment the
%     \makeapproval command.
%     it should come after dedication, if dedication is
%     present. Otherwise it is the first page after title page.

\makecertificate

\makeapproval



%--------------------------------------------------------------------%
% COPYRIGHT PAGE
%   - To include a copyright page use \copyrightpage
% \copyrightpage

%--------------------------------------------------------------------%
% ABSTRACT

\thispagestyle{empty}
%\input{declaration.tex}
\makedeclaration
\begin{titlepage}
\thispagestyle{empty}
%\pagestyle{empty}
\begin{center}
\begin{LARGE}
\bf {Abstract}
\end{LARGE}

\end{center}
\large
In 2014, the Electricity Consumption per capita in India was 805.6KWh which is equivalent to 637.43 kg of oil per capita. Over 58 percent of this electricity is produced from non renewable sources of energy. Our dependence on production of energy from non renewable sources of energy makes India both a major greenhouse gas emitter and one of the most vulnerable countries in the world to projected climate change. The country is already experiencing changes in climate and the impacts of climate change, including water stress, heat waves and drought, severe storms and flooding, and associated negative consequences on health and livelihoods. It is imperative that India rapidly adopts renewable sources of energy like solar and wind. But in addition to that it is also the responsibility of the Indian people to monitor their energy consumption and reduce their carbon footprint. 
The Smart Energy Monitor helps the Indian consumer to reduce and monitor their household energy consumption by providing insights to consumption of electricity by individual electrical appliances. The Smart Energy Monitor  connects directly to your electricity panel and uses a mobile app to tell you what devices and appliances are drawing power and when. The monitor listens to the electronic signature of each device and uses algorithms to identify them and monitor their power consumption. It also presents real-time and historical usage for each device. It will help the consumer track energy inefficient appliances and also their monthly usage. From these insights the consumer can reduce their electricity consumption thereby reducing their carbon footprint. 





\end{titlepage}


%--------------------------------------------------------------------%
% CONTENTS, TABLES, FIGURES
%\tableofcontents
%\listoftables
%\listoffigures

%--------------------------------------------------------------------%
% NOMENCLATURE
%\begin{nomenclature}
%\begin{description}
%\item{\makebox[0.75in][l]{$C_1$}} Constant 1
%
%\item{\makebox[0.75in][l]{$V$}}    Voltage 
%
%\item{\makebox[0.75in][l]{\$}}     US Dollars
%\end{description}
%\end{nomenclature}
%
%\cleardoublepage\pagenumbering{arabic} % Make the page numbers Arabic (1, 2, etc)
   %%Make changes in this file for title of the project, abstract and details of the author
\begin{acknowledgments}

\thispagestyle{empty}

\noindent We have great pleasure in presenting the report on {\bf "\projecttitle"}. I take this opportunity to express my sincere thanks towards the guide Prof. Jayen Modi, C.R.C.E, Bandra (W), Mumbai, for providing the technical guidelines, and the suggestions regarding the line of this work. We enjoyed discussing the work progress with him during our visits to department.\\

\noindent We thank Dr. Deepak V. Bhoir, Head of Electronics Dept., Principal and the management of C.R.C.E., Mumbai for encouragement and providing necessary infrastructure for pursuing the project.\\

\noindent We also thank all non-teaching staff for their valuable support, to complete our project.
\par
%

%\hfill 

\end{acknowledgments}
%I thank the many people who have done lots of nice things for me.



\newpage
\tableofcontents
\listoffigures
%\listoftables
\printglossaries
\addcontentsline{toc}{chapter}{Glossary}
%\addcontentsline{toc}{chapter}{List of Abbreviations}
\printnomenclature[2in]
\cleardoublepage\pagenumbering{arabic}
\pagestyle{plain}
\chapter{Introduction}
{
The Smart Energy Monitor is based on the concept of Non Intrusive Load Monitoring (NILM) which is a process for analyzing changes in the voltage and current going into a house and deducing what appliances are used in the house as well as their individual energy consumption. Electric meters with NILM technology are used by utility companies to survey the specific uses of electric power in different homes. NILM is considered a low-cost alternative to attaching individual monitors on each appliance. 

The system can measure both reactive power and real power. Hence two appliances with the same total power draw can be distinguished by differences in their complex impedance. For example, a refrigerator electric motor and a pure resistive heater can be distinguished in part because the electric motor has significant changes in reactive power when it turns on and off, whereas the heater has almost none.

NILM systems can also identify appliances with a series of individual changes in power draw. These appliances are modeled as finite state machines. A dishwasher, for example, has heaters and motors that turn on and off during a typical dish washing cycle. These will be identified as clusters, and power draw for the entire cluster will be recorded. Hence ``dishwasher'' power draw can be identified as opposed to ``resistor heating unit'' and ``electric motor''.
Thus designing a energy monitoring unit using this NLIM has many benefits. The overview of the system is as follow:
\begin{figure}[H]
	\includegraphics[scale=0.5]{introblockdia} % first figure itself
	\caption{Basic Block diagram of the System}
	\label{blck}
\end{figure}
}
%--------------------------------------------------------------------
% \LaTeX users - skip the section below
%-------------------------------------------------------------------


%--------------------------------------------------------------------------

\section {Motivation}
How many planets would it take to support our lifestyle? As blunt as it may sound, the truth stands unchanged, staring at the face of the unknown future of the whole planet. It didn't take us long to open our hearts (and homes) to the amazing changes that technology brought into our lives. Amidst the ease and comfort, everything else seems to be collateral damage to us now. One such commodity is electricity.Electricity consumption rates from non renewable sources of energy is increasing at an alarming rate by the hour. Our increasing dependency on these replenishing  sources calls for innovative ways to tap on our consumption rates, preferably on a daily basis. Electrical appliances used on a daily basis are monitored ambiguously, hence an increase in the prices may not be found. 
The outcome of this project, the Smart Energy Monitor is aimed to help the average Indian consumer monitor their appliance usage on a daily basis hence keep track of their consumption.

\section{Objectives}
\begin{enumerate}
	\item To disaggregate load appliances using minimum hardware & efficient and accurate classification algorithms.
	\item To provide the user with Real-time monitoring and alerts on their smartphone.
	\item To track the power consumption of individual load appliances and provide the user with an estimate of their monthly electricity bill.
\end{enumerate}
\newcommand{\tab}{\hspace*{2em}}
\chapter{Literature Review} 

\section{Power Measurement}
{
Most current sensors can measure current in two directions.  this means is that if we sample fast enough and long enough,  we sure to find the peak in one direction and the peak in another direction. With both peaks known, it is a matter of knowing the shape of the waveform to calculate the current. In the case of line or mains power, we know that waveform to be a Sine wave. Hence the expression of AC current  will be in a value known as RMS. 

\begin{figure}[H]
	\includegraphics[scale=1]{vrms} % first figure itself
	\caption{Measurement of Vrms}
	\label{blck}
\end{figure}

Conversion for a sine wave with a zero volt offset (like that in mains or line power) is performed as follows:

\begin{enumerate}
	\item Find the peak to peak voltage  ( Volts Peak to Peak )
	\item Divide the peak to peak voltage by two to get peak voltage (Volts Peak)
	\item Multiply the peak voltage by 0.707 to yield rms volts (Volts RMS)
\end{enumerate}

Having Calculated RMS voltage,  is simply a matter of multiplying by the scale factor of the particular current sensor to yield the RMS value of the current being measured which can then be multiplied by the AC Voltage value in order to give the value of the Total Power Drawn.

\begin{equation}
I_r_m_s = \frac{V_p_p \times 0.707 \times Scale Factor}{2} 
\end{equation}

\begin{equation}
P = I_r_m_s \times 230
\end{equation}

}



\section{Disaggregation Algorithms}
Most NLIM systems provide implementations of two common benchmark
disaggregation algorithms: Steady State Analysis and Combinatorial Optimisation(CO).

\subsection{Steady State Analysis}
The NILM methods based on steady-state analysis make use of steady-state features that are derived
under the steady-state operation of the appliances. Real power (P) and Reactive power (Q) are two of the
most commonly used steady state signatures in NILM  for tracking On/Off operation of appliances.
The real power is the amount of energy consumed by an appliance during its operation. If the load is
purely resistive then the current and voltage waveforms will always be in phase and there will be no
reactive energy. For a purely reactive load the phase shift will be 90 degrees, and there will be no transfer of real
power. On the other hand, due to inductive and capacitive elements of the load, there is always a phase
shift between current and voltage waveforms that generates or consumes a reactive power respectively.

\begin{figure}[H]
	\includegraphics[scale=1]{steadystate} % first figure itself
	\caption{Load Distribution in PQ Plane and Current draw of Linear vs Non Linear Loads}
	\label{blck}
\end{figure}

Researchers have tried to disaggregate load using real power as a single feature and found
out that high-power appliances with distinctive power draw characteristics such as electrical heaters and
water pumps can be easily identified from the aggregated measurements. However this method does
not take into account appliances with similar power draw characteristics. In addition, simultaneous state
transitions of appliances leads to erroneous results. In order to address some of these issues, high power appliances can easily be differentiated
by analyzing the step changes in real and reactive power features.

\subsection{Combinational Optimization(CO)}

Combinational Optimization finds the optimal combination of appliance states, which minimizes the difference between the sum of the predicted appliance power and the observed aggregate power, subject to a set of appliance models. Since each time slice is considered as a separate optimisation problem, each time slice is assumed to be independent.
The complexity of disaggregation for T time slices is: 

\begin{equation}
N_C_o_m_b_i_n_a_t_i_o_n_s = K ^ N
\end{equation}

where N is the number of appliances and
K is the number of appliance states.

Since the complexity of CO is exponential in the number of appliances, the approach is only computationally tractable for a small number of modelled appliances. The error can be minimized by choosing the combination whose calculated power draw is the closest to the measured power drawn.


\section{Public Datasets}

Apart from a common evaluation metric there is also a lack of reference dataset on which the
performance of algorithm can be compared. It is quite obvious that the output of the load disaggregation
algorithm is dependent on the source data, which often varies either due to difference in the number
and type of appliances used in the experiment or due to the hardware used to extract the load signatures. In order to draw meaningful performance comparison of various NILM techniques,
the availability of common datasets is critical. Motivated by this, recently the Reference Energy Disaggregation Data Set (REDD) and the Building-Level fUlly labeled Electricity Disaggregation dataset (BLUED) have been made publicly available in order to facilitate the researchers in the development and evaluation of new load disaggregation algorithms. The datasets contain high-frequency and low-frequency household power measurements primarily for the evaluation of steady-state as well as transient state NILM methods.

\subsection{Reference Energy Disaggregation Dataset (REDD)}
The data contains power consumption from real
homes, for the whole house as well as for each individual circuit in
the house (labeled by the main type of appliance on that circuit).
The data is intended for use in developing disaggregation methods, which
can predict, from only the whole-home signal, which devices are being
used. The REDD data set contains two main types of home electricity data:
high-frequency current/voltage waveform data of the two power mains
(as well as the voltage signal for a single phase), and
lower-frequency power data including the mains and individual,
labeled circuits in the house. 

\subsection{Building-Level fUlly labeled Electricity Disaggregation (BLUED) Dataset}
The BLUED dataset consists of voltage and current measurements for a single family residence in the United States, sampled at 12 kHz for a whole week. Every state transition of each appliance in the home during this time was labeled and time-stamped, providing the necessary ground truth for the evaluation of event-based algorithms. 















 
 
 
 
 
 
 
 
 
 
 

 
 
 
 
 
 
 


%\input{modifications.tex}
\chapter{Report on present investigations:}
\section{Raspberry Pi3}{
	\begin{figure}[H]
		\includegraphics[width=0.9\textwidth]{images.jpg} % first figure itself
		\caption{Raspberry Pi 3}
		\label{trans}
	\end{figure}
	The Raspberry Pi 3 is the third generation Raspberry Pi. This powerful
    credit-card sized single board computer can be used for many applications
    and supersedes the Raspberry Pi 2 Model. Whilst maintaining the popular board format the Raspberry Pi 3 Model
    B brings you a more powerful processer, 10x faster than the first generation
    Raspberry Pi. Additionally it adds wireless LAN & Bluetooth connectivity
    making it the ideal solution for powerful connected designs. It has a HDMI (rev 1.3 & 1.4 Composite RCA (PAL and NTSC).It uses Broadcom BCM2387 chipset with a
    1.2GHz Quad-Core ARM Cortex-A53 and 802.11 b/g/n Wireless LAN and Bluetooth 4.1 (Bluetooth Classic and LE). It also has a 1GB LPDDR2 memory. Operating System Boots from Micro SD card, running a version of the Linux operating system or
    Windows 10 IoT.
	%\begin{equation}
	%	C = 0.7 \times I /(\Delta V \times F)
	%\end{equation}
	
}
\section{ACS712 Module Current Sensor}{
	%\subsection{Calculate the Maximum Switch Current}
	\begin{figure}[H]
	    \includegraphics[width=0.9\textwidth]{PinDiagrams.jpg}
	    \caption{ACS712}
	    \label{fig:my_label}
	\end{figure}
	The Allegro™ ACS712 provides economical and precise solutions for AC or DC current sensing in industrial, commercial, and communications systems.The device consists of a precise, low-offset, linear Hall circuit with a copper conduction path located near the surface of the die. Applied current flowing through this copper conduction path generates a magnetic field which the Hall IC converts into a proportional voltage. Device accuracy is optimized through the close proximity of the magnetic signal to the Hall transducer. A precise, proportional voltage is provided by the low-offset, chopper-stabilized BiCMOS Hall IC, which is programmed for accuracy after packaging. The internal resistance of this conductive path is 1.2 mΩ typical, providing low power loss. The thickness of the copper conductor allows survival of the device at up to 5× overcurrent conditions. The terminals of the conductive path are electrically isolated from the signal leads (pins 5 through 8). This allows the ACS712 to be used in applications requiring electrical isolation without the use of opto-isolators or other costly isolation techniques. It requires a 4.5-5.5V power supply. It takes input current of 0-30A and produces output voltage of 2.5-5V.
%	\begin{equation}\label{eq1}
%	D = \frac{V_{out}}{V_{IN(max)} \times \eta}
%	\end{equation}
%	Where,\\
%	$V_{IN(max)}$ = maximum input voltage\\
%	\subsection{Inductor Selection}
%	\begin{equation}
%	L= \frac{Vout \times (Vin-Vout)}{ \Delta I_L \times fs\times vin} 
%	\end{equation}
%	Where,\\
%	$V_{IN}$ = typical input voltage\\
%	$V_{OUT}$ = desired output voltage\\
%	$f_S$ = minimum switching frequency of the converter\\
%	$\Delta I_L$ = estimated inductor ripple current\\
%	$\Delta I_L$ = estimated inductor ripple current\\
%	$I_{OUT(max)}$ = maximum output current necessary in the application\\
	
%	\subsection{Rectifier Diode Selection}
}
\section{ADS1115 Analog to Digital Converter}{
%\subsection{Power Supply Unit of the Peltier TEC1-12706}
\begin{figure}[H]
	\includegraphics[width=0.9\textwidth]{fbd_sbas444c.jpg} % first figure itself
	\caption{ADC circuit}
	\label{ADC}
\end{figure}
The ADS1115 device is a precision, low-power, 16-bit, I2C-compatible, analog-to-digital converters (ADCs) offered in an ultra-small, leadless, X2QFN-10 package, and a VSSOP-10 package. The ADS111x devices incorporate a low-drift voltage reference and an oscillator. The ADS1114 and ADS1115 also incorporate a programmable gain amplifier (PGA) and a digital comparator. These features, along with a wide operating supply range, make the ADS111x well suited for power- and space-constrained, sensor measurement applications.

The ADS1115 perform conversions at data rates up to 860 samples per second (SPS). The PGA offers input ranges from ±256 mV to ±6.144 V, allowing precise large- and small-signal measurements. The ADS1115 features an input multiplexer (MUX) that allows two differential or four single-ended input measurements. Use the digital comparator for under- and overvoltage detection.

The ADS111x operate in either continuous-conversion mode or single-shot mode. The devices are automatically powered down after one conversion in single-shot mode; therefore, power consumption is significantly reduced during idle periods. It has a resolution of 16 bits, sample rate of 860SPS, 4 Input channel, I2C, Interface, 2-5.5V Power Supply,

}
\section{User Interface}
{The user interface uses a Python Micro-Framework called Flask which is used to build web applications. The Flask framework encodes the real time Python Variables into a JavaScript Object Notation(JSON) string. The JSON string is then read into the HTML file using Asynchronous JavaScript And XML(AJAX) requests. These requests are periodically refreshed using the Auto-Refresh function in AJAX.}
\begin{figure}[H]
	\includegraphics[width=0.9\textwidth]{Demo1.PNG} % first figure itself
	\caption{HTML-Based UI}
	\label{UI}
\end{figure}
{The user interface displays various system parameters such as Total Power being Consumed, Active Devices and the Real Time Clock. Additional parameters such as Estimated Monthly Electricity bill can be added in the future.}

\section{Classification Algorithm (KNN)}{
In machine learning and statistics, classification is the problem of identifying to which of a set of categories (sub-populations) a new observation belongs, on the basis of a training set of data containing observations (or instances) whose category membership is known. An example would be assigning a given email into "spam" or "non-spam" classes or assigning a diagnosis to a given patient as described by observed characteristics of the patient (gender, blood pressure, presence or absence of certain symptoms, etc.). Classification is an example of pattern recognition.

In the terminology of machine learning, classification is considered an instance of supervised learning, i.e. learning where a training set of correctly identified observations is available. The corresponding unsupervised procedure is known as clustering, and involves grouping data into categories based on some measure of inherent similarity or distance.

Often, the individual observations are analyzed into a set of quantifiable properties, known variously as explanatory variables or features. These properties may variously be categorical (e.g. "A", "B", "AB" or "O", for blood type), ordinal (e.g. "large", "medium" or "small"), integer-valued (e.g. the number of occurrences of a particular word in an email) or real-valued (e.g. a measurement of blood pressure). Other classifiers work by comparing observations to previous observations by means of a similarity or distance function.

An algorithm that implements classification, especially in a concrete implementation, is known as a classifier. The term "classifier" sometimes also refers to the mathematical function, implemented by a classification algorithm, that maps input data to a category.
 
KNN ( k Nearest Neighbours)
In pattern recognition, the k-nearest neighbors algorithm (k-NN) is a non-parametric method used for classification and regression. In both cases, the input consists of the k closest training examples in the feature space. The output depends on whether k-NN is used for classification or regression:

In k-NN classification, the output is a class membership. An object is classified by a majority vote of its neighbors, with the object being assigned to the class most common among its k nearest neighbors (k is a positive integer, typically small). If k = 1, then the object is simply assigned to the class of that single nearest neighbor.
In k-NN regression, the output is the property value for the object. This value is the average of the values of its k nearest neighbors.
k-NN is a type of instance-based learning, or lazy learning, where the function is only approximated locally and all computation is deferred until classification. The k-NN algorithm is among the simplest of all machine learning algorithms.

Both for classification and regression, it can be useful to assign weight to the contributions of the neighbors, so that the nearer neighbors contribute more to the average than the more distant ones. For example, a common weighting scheme consists in giving each neighbor a weight of 1/d, where d is the distance to the neighbor.

The neighbors are taken from a set of objects for which the class (for k-NN classification) or the object property value (for k-NN regression) is known. This can be thought of as the training set for the algorithm, though no explicit training step is required.

A peculiarity of the k-NN algorithm is that it is sensitive to the local structure of the data.[citation needed] The algorithm is not to be confused with k-means, another popular machine learning technique. 
 Following example would help us to understand better :
 The test sample (green circle) should be classified either to the first class of blue squares or to the second class of red triangles. If k = 3 (solid line circle) it is assigned to the second class because there are 2 triangles and only 1 square inside the inner circle. If k = 5 (dashed line circle) it is assigned to the first class (3 squares vs. 2 triangles inside the outer circle). 
\begin{figure}
    \includegraphics[width=0.9\textwidth]{Knn.png}
    \caption{K Nearest Neighbour}
    \label{fig:my_label}
\end{figure}
}
\chapter{Results}
\section{Observations}
{
We successfully measure changes in the current drawn by the load appliances when their states change. By measuring the Phase Angle we can get Active Power and Reactive Power which further improves the accuracy of the system. We created an dataset which contains labelled data which is passed to our classification algorithm. The Naive Bayes algorithm uses the mean & standard deviation of the input parameters to determine the active devices.  }

\section{Drawback }
The user must wait for a period of 0.5s to 1s before turning ON/OFF any load appliance.  

The Smart Energy Monitor is not a plug and play device. Since it deals with high levels of voltage & current, it must be installed within the household by a professional electrician.

\section{Future Scope}
\begin{enumerate}
    \item The classification accuracy can be improved through pattern analysis by monitoring the shape of the VI trajectory. Shape features such as area under the VI curve as well as peak of segments can be further analyzed.
    \item Analysis of Steady  State Voltage Noise such as EMI signatures can improve the detection of motor based devices like Fans, Food Mixers and Washing Machines.
    \item Unsupervised learning methods such as Artificial Neural Networks(ANN) and Hidden Markov Model(HMM) have shown to perform well for the task of load disaggregation due to their ability to incorporate in their learning, temporal as well as appliance state transition information.
\end{enumerate}
%\input{test.tex}
%\input{system.tex}
%\input{implement.tex}
%\input{summary.tex}
%\input{intro.tex}
%\input{Chapter2.tex}
%\input{Stability.tex}
%\input{StateSpace.tex}
%\input{CircuitExample.tex}
%\input{SolvedEXP.tex}
%\input{FRSS.tex}
\chapter{\Large{Conclusion}} \label{con}
The system is able to calculate the power consumption in a room with up to a 5 percent margin of error. Using this information, assuming the devices are varied, it can determine which devices are being used. This in turn will also help us to figure out how much of the total power each device is consuming. By using one sensor in every room, an entire household can be monitored the same way.
The system cannot as of yet be used for a large number of devices in one room. To do this we must include a voltage sensor to reduce reading fluctuations. Also, a better classification algorithm is required that doesn't only depend on the power output of the sensors.
In the next semester we will add these secondary factors into our KNN classification algorithm. We will also use the event detection to add more features to the user interface. Finally, we will attempt to create an app that will dynamically be updated with the UI.
\begin{thebibliography}{9}
\bibitem{opennlim} 
Nipun Batra, Jack Kelly, Oliver Parson, Haimonti Dutta, William Knottenbelt, Alex Rogers, Amarjeet Singh and Mani Srivastava
\textit{NILMTK: An Open Source Toolkit for Non-intrusive Load
            Monitoring}. 

\bibitem{nlimsensors} 
Ahmed Zoha, Alexander Gluhak, Muhammad Ali Imran, Sutharshan Rajasegarar
\textit{Non-Intrusive Load Monitoring Approaches for Disaggregated
    Energy Sensing: A Survey}. 
\\\texttt{www.mdpi.com/1424-8220/12/12/16838/pdf}

\bibitem{henryscurrent} 
Henry's Bench
\textit{How to Measure AC Current with an Arduino and an ASC712}. 
\\\texttt{http://henrysbench.capnfatz.com/henrys-bench/arduino-current-measurements/acs712-arduino-ac-current-tutorial}

\bibitem{watty} 
Watty - 
\textit{A simple way to keep track of what goes on at home}. 
\\\texttt{https://watty.io/}
            
\end{thebibliography}
%\renewcommand\bibname{References}
%\addcontentsline{toc}{chapter}{Appendix}
%\appendix
%\input{main.tex}
%\insertcode{"scilab.sci"}{A sample Scilab code} % The first argument is the script location/filename and the second is a caption for the listing
%\nocite{*}
%\renewcommand\bibname{Bibliography}
%\bibliographystyle{IEEEtran}
%\bibliography{project-ref}
%\addcontentsline{toc}{chapter}{Bibliography}
%\setcounter{toc}{-1}


\end{document}
